\chapter{Conceptual Overview}
\label{section:conceptual_overview}
This section introduces the main concepts which you will need to have
some grasp of in order to use \tao\ at anything but the simplest level.
Topics covered include \tao's \emph{cellular material}; \emph{instruments}; 
\emph{devices}; \emph{access points}; \emph{parameters}; and \emph{pitches}.

\section{\tao's Cellular Material}
\label{section:cellular_material}

\tao\ is based around the notion of building complex vibrating structures
from simpler acoustic building blocks. In order to realise this goal 
a general purpose adaptable acoustic material is provided. The material
consists of point masses arranged in a regular grid pattern and connected
together with springs. The overall structure of the material is shown in
figure \ref{fig:cells}. 

\begin{figure}[htb]
  \begin{Label}{fig:cells}
    \begin{center}
    \Image{cells}{height=5cm}{gif}
    \end{center}
    \caption{A small portion of \tao's cellular acoustic material}
  \end{Label}
\end{figure}

Each point mass or \Term{cell} maintains a set of state variables for
its position, velocity, mass, etc., and the overall state is updated in
discrete time steps or \Term{ticks} according to rules which take into
account a cell's own state and the states of its immediate neighbours.
More specifically a cell's spring connections to its neighbours exert
forces on the cell and from these forces Newton's laws of motion can
be used to calculate the acceleration and velocity of the cell.

Note that the cells are constrained to have one degree of freedom (in the
direction of the $z$ axis). This has two practical advantages:

\begin{itemize}
\item It makes the calculations involved in animating the material simpler.
\item It also makes it a simple matter to generate time varying waveforms
from the vibrations in the material since all cells will vibrate about a
fixed zero reference point at $z=0$ so long as at least one cell
is fixed at that position.
\end{itemize}

One question which has often been asked in relation to \tao's cellular
material is: `have you experimented with 3D blocks of material?'. The answer
to this question is no, for the following reasons. Firstly, the computational
expense of such instruments would be prohibitive, and secondly, although the
restriction of working with 2D instruments may at first seem like a limitation,
in practice it doesn't significantly affect \tao's ability to produce an
wide variety of interesting sounds.

\subsection{Cell Attributes}
\label{section:cell_attributes}
The most important \Term{attributes} maintained by each cell are its
\Attr{mass}, \Attr{position}, \Attr{velocity}, 
\Attr{force} and lastly \Attr{velocityMultiplier}. Not
surprisingly these are used to keep track of the cell's state of motion
and the forces acting upon it. The \Class{Cell} class is unusual in
that it is seldom dealt with directly. However we will see that many
of \tao's other classes have attributes which are often accessed
and set via \Term{methods}

The \Attr{velocityMultiplier} attribute is quite important. It holds
a value in the range [0..1] and the velocity of the cell is multiplied
by this value on each tick. This has the effect of dissipating energy (if the
value is $<$ 1) and thus damping the vibrations of the cell. Each cell
can have a different value for this attribute leading to non-uniform
damping of the material and this feature is a very important tool for
controlling the vibrational characteristics of the material as we will
see later on in this manual. 

For example with a simple string-like instrument consisting of a single row
of cells linked together with springs, damping small regions at either end
of the string more highly than the rest of the cells
causes the higher frequency vibrations to die away more quickly than the
lower ones. This leads to a more natural string-like spectral decay in the
sounds produced by the instrument, whereas a string with uniform damping
exhibits no significant change in the distribution of spectral energy as
the sound evolves. 

It is probably safe to say that all physically produced sounds
exhibit some kind of spectral evolution, and as a general rule
non-uniform damping always produces more interesting sounds from
\tao\ instruments than uniform damping.

The other main attributes that each cell maintains are pointers to its
neighbours \Attr{north}, \Attr{east}, \Attr{south},
\Attr{west}, \Attr{neast}, \Attr{nwest}, \Attr{seast}
and \Attr{swest}. Each pointer indicates to a particular cell that
it is connected to a neighbouring cell via a virtual spring. Similarly the
neighbouring cell will reciprocate by keeping a pointer to the first. You do
not need to deal with these pointers directly as an end-user \tao\ but it
is worth knowing that they are there. 

The final attribute is used to store other aspects of the cell's state,
such as whether it is locked or free to move. This attribute is called
\Attr{mode}.

\subsection{The Emergent Behaviour of the Material}
Having described the microscopic structure of the cellular material
it is now time to say something about its macroscopic behaviour.
One of the most appealing features of computer models in which many
simple elements interact on a local basis according to well-defined
rules is that they often exhibit interesting \Term{emergent behaviour}.
In \tao's case the material appears to behave (not surpisingly) like an
continuous elastic sheet when viewed from a distance.

Figure \ref{fig:circle_example} shows a typical piece of \tao's material,
in this case a circular sheet which has had a short impulse applied at a
single point. The image is a snapshot taken some short time interval later
and clearly shows wavefronts spreading out from the point of impact and
also reflecting off of the boundary of the object.

\begin{figure}[htb]
  \begin{Label}{fig:circle_example}
    \begin{center}
    \Image{circle_example}{height=6cm}{gif}
    \end{center}
    \caption{Screenshot of typical \tao\ instrument}
  \end{Label}
\end{figure}

In the graphical representation used the individual cells and springs
are not visible. Instead what we see is a wireframe representation in which
each line represents a single row of cells. Note how smoothly contoured
the waves are, giving the impression that the material is 
continuous and elastic in behaviour.

\subsection{Generating Sound Output from the Material}
In order to generate output waveforms the vibrations of the material
are `listened' to directly. In order to achieve this a `sensor' is placed on the
surface of the material at a chosen point and a time-varying trace of
the motion of the point with respect to the $z$ axis is written to an
output file. This numerical data can then be played back as digital
audio samples.

Output signals may be derived from mathematical expressions
involving more than one point on an instrument. In addition, the
positions of the points from which the signals are generated may be moved
around under algorithmic control. These are just two of the tools at
your disposal for creating interesting dynamically evolving sounds.

\section{Instruments and Devices}
\label{section:instruments_and_devices}
Although the cellular material is actually composed of many hundreds of
individual objects representing the cells and springs, from the user's
point of view a higher level of abstraction is provided for creating
and interacting with pieces of the material. This abstraction comes in
the form of \emph{instruments}\index{instrument} and
\emph{devices}\index{device}.

\tao\ provides a set of classes for creating primitive acoustic
building blocks. These are derived from a generic \ClassIndex{Instrument}
class and include \Instr{String}, \Instr{Rectangle},
\Instr{Circle}, \Instr{Ellipse} and \Instr{Triangle}
\texonly{\index{instrument!building blocks}}. The \Instr{String}
class creates a single line of cells and springs whilst the other
classes create 2-D sheets of material in a variety of shapes. 
Using these instrument classes you don't need to worry about
creating the individual cells or springs which link them together
as this is taken care of for you. 

It should be mentioned here that in the rest of this document the
generic term \Term{instrument} is used to refer both to these simple
building blocks and also more complex arrangements in which several pieces of
material are coupled together. In the latter case the term
\Term{compound instrument} is used. 

A number of other classes derived from a common \ClassIndex{Device} class
are also provided. Devices are objects which:

\begin{itemize}
\item
allow the primitive instruments listed above to be coupled together into more
interesting \Term{compound instruments};
\item 
provide the means for applying external excitations to the instruments;
\item
enable sound output to be generated by `listening' to points on instruments,
writing the resulting time varying waveforms to sound files.
\end{itemize}

Devices which are available in the current version of
\tao\ include bows, hammers, connectors, stops and outputs.
The purpose of each of each type of device is explained in the
following sections. But before moving on another fundamental
object class needs to be introduced -- the \Term{access point}.
Whenever a device interacts with an instrument in some way an
access point is involved. You don't need to worry too much
about how to use them at the moment but they are described
later on in section \ref{section:access_points}.

\subsection*{The Bow Device}
\label{section:bow_device}
\index{Bow device}
The bow device provides the user with a model of the interaction
between a bow and an instrument. It works by mathematically modelling the
static and dynamical frictional forces which occur between the bow and the
instrument. Each bow device has the following user accessible
attributes:

\begin{description}
\item[\Attr{velocity}] -- the current velocity of the bow.
\item[\Attr{force}] -- the downward force applied to the bow. 
\end{description}

The bowing model used in this device is based loosely upon a model
developed by Woodhouse.

\subsection*{The Hammer Device}
\label{section:hammer_device}
\index{Hammer device}
The Hammer device provides a generalised mechanism for producing
percussive sounds. A Hammer device has the following attributes:

\begin{description}
\item[\Attr{mass}] -- the mass of the hammer.
\item[\Attr{height}] -- the initial height from which the hammer is dropped.
\item[\Attr{position}] -- the current height of the hammer.
\item[\Attr{initVelocity}] -- the initial velocity of the hammer.
\item[\Attr{velocity}] -- the current velocity of the hammer.
\item[\Attr{maxImpacts}] -- the maximum number of the impacts with the
target instrument.
\item[\Attr{numImpacts}] -- the number of impacts with the target instrument
since the hammer was dropped.
\item[\Attr{gravity}] -- gravitational acceleration acting upon the hammer.
\item[\Attr{damping}] -- degree of damping applied to the hammer.
\item[\Attr{hardness}] -- how hard the impact surface of the hammer is.
\end{description}

All of these parameters may be altered under algorithmic control except the
number of impacts which is read only. 

\subsection*{The Connector Device}
\label{section:connector_device}
\index{Connector device}
The Connector device provides a flexible mechanism for coupling 
instrumental components together and coupling points on instrumental 
components to fixed \Term{anchor} points. It does so by installing 
springs between the
access points or anchors specified. Anchor points may be numerical constants
(usually 0.0) or arbitrary expressions, the latter being useful for driving
an instrument with an external signal (external to the instrument, not \tao\ 
itself).

The technique of connecting an access point to a fixed anchor point
is sometimes useful for restricting the amplitude of vibrations at certain
points on an instrument. For example a component might have too many low
frequency partials making the sound too bottom-heavy in which case various
points on the instrument can be connected to zero anchor points allowing only
the higher partials to continue vibrating. An extra attribute allows the
strength of the installed spring to be set. This takes a value in the range
[0..1] normally although higher values may work.

One of the most powerful features of \tao\ lies in the fact that the
coordinates specifying the position of an access points do not have to be
constant. They can be time-varying values derived from expressions in the
score. This leads to the ability to create instruments which morph
structurally as they are being played. The Connector device is
therefore one of the most important provided by \tao\ since it enables
complex, dynamically evolving instruments to be constructed.

\subsection*{The Output Device}
\label{output_device}
\index{Output device}
The Output device provides a general means for writing floating point
samples to output files. The samples are initially un-normalised but the 
resulting data files may be normalised and converted into WAV format sound
files with a separate program \Prog{tao2wav}. The minor inconvenience of
having to convert \tao\ output files into a more usable format as a separate
post-processing stage is imposed for good reason. Firstly it does have the
advantage that the user doesn't have to worry about sound samples going out
of range. Also, with a model which is heavily based around floating point
calculations and physically vibrating entities it is impossible to know the
amplitude of the vibrations in advance.

The present version of \tao\ only allows one and two channel output files but
this will be changed in future releases to allow for arbitrary numbers of
channels.

\subsection*{The Stop Device}
\label{stop_device}
\index{Stop device}
The Stop device provides a mechanism for producing specific pitches from a
string by stopping or fretting  it, i.e. temporarily changing its effective
vibrating length. The Stop device is not an entirely accurate model of
how a stopped string behaves and will probably be refined in future releases.
It currently works by installing a Connector device between
the specified access point and a fixed anchor position at 0.0. This restricts
the vibrations of the string at the access point. In addition there is
a `damping' attribute which causes a small region around the access point to
be more highly damped than the rest of the string. This causes the string to
almost stand still at the access point, which is the desired effect.


\subsection{The Information Needed to Create an Instrument}

When a new instrument is created three pieces of information are required
from the user in the case of rectangular and elliptical sheets and two in
the case of strings and circular sheets. These are:

\begin{itemize}
\item A frequency or pitch specifying the dimensions of the instrument
in the $x$ direction
\item A frequency or pitch specifying the dimensions of the instrument
in the $y$ direction (for rectangular and elliptical sheets only)
\item A decay time
\end{itemize}

The dimensions of a new instrument are always described in terms of
pitches or frequencies rather than physical units such as metres or
millimetres. For example when creating a new string the pitch specified
is used to determine the length of the string.
You may be thinking `what about the tension in the string?', but in
\tao's cellular material the tension is fixed at an optimum value which
gives the best frequency response, so the only factor affecting the
pitch of a string is its length and vice versa.

In the case of a rectangular sheet two pitches or frequencies are
required. The first relates to the time taken for a wavefront to make
the round trip from the left hand side of the sheet to the right and
back again (travelling in the $x$ direction). The second relates
to the time taken for a wavefront to make the round trip from the bottom
to the top and back again (travelling in the $y$ direction).

\hierindex{instrument!information needed to create}
\begin{figure}[hbt]
  \begin{Label}{fig:xandyfreqs}
    \begin{center}
    \Image{xandyfreqs}{height=7cm}{gif}
    \end{center}
    \caption{Relationship between x and y frequency and an instrument's size}
  \end{Label}
\end{figure}

The precise relationship between the frequencies and dimensions of the
various instrument types is illustrated by figure \ref{fig:xandyfreqs}.
The constant \verb|Hz2CellConst| is defined by \tao\ and is used to
translate a frequency into the appropriate number of cells required
to produce that frequency of vibration.

In order to explain this more clearly figure \ref{fig:circle_waves}
shows four snapshots of a circular sheet having had a short impulse
applied at its centre. The white arrows
indicate part of the wavefront traveling across the sheet in the 
$x$ direction, reflecting back off the boundary and eventually
ending up back at the starting point again. Once the wavefront has
traveled all the way to the other side of the sheet and then back
to the centre again it has made one round trip. The speed of wave
propagation is fixed for \tao's material so the \verb|Hz2CellConst|
constant can be used to calculate how many cells it takes to produce
the correct period $T$, and hence frequency $1/T$.

\begin{figure}[htb]
  \begin{Label}{fig:circle_waves}
    \begin{center}
    \Image{circle_waves}{height=7cm}{gif}
    \end{center}
    \caption{Round trip of a wavefront in a circular sheet}
  \end{Label}
\end{figure}


The final piece of information required by all new instruments is a
decay time, the time taken for vibrations to naturally die away
when the instrument is excited in some way and then left to vibrate
freely. The value given is used to calculate a default value
for the \verb|velocityMultiplier| attribute of each cell. This initially
gives the instrument uniform damping although this uniform behaviour
can be subsequently altered by damping local regions of the instrument.
More of this technique in section [TO DO: WRITE THIS SECTION].

\section{Access Points}
\label{section:access_points}
In order for \tao\ to provide an interface between instruments and devices
another key object is needed: the \Term{access point}. Access points
allow forces to be applied to the material and also for the physical
attributes of the material to be read off at \emph{any}\/ point, not just
at the discrete set of points where cells exist. This interpolation facility
is one of the most important as it overcomes some of the limitations associated
with the material being discrete in nature. An example might be trying to
simulate a string being stopped to a particular pitch. Without the ability
to interpolate the position at which the stop is applied it might not be
possible to achieve the desired pitch precisely.

\begin{figure}[htb]
  \begin{Label}{fig:instrument_coord_system}
    \begin{center}
    \Image{instrument_coord_system}{height=8cm}{gif}
    \end{center}
    \caption{The Instrument Coordinate System}
  \end{Label}
\end{figure}

In short, all interactions between \tao's cellular material and the outside
world take place via access points. This is probably a good place to 
introduce the notion of the
\emph{instrument coordinate system}\hierindex{instrument!coordinate system}.
This coordinate system allows points in the range $(0..1, 0..1)$ to be
specified.
All instruments, regardless of their shape and size use the same 
normalised coordinate system as illustrated in figure
\ref{fig:instrument_coord_system}. Of course for some instruments
such as the circular sheet depicted some points (such as point \textbf{a})
will be invalid. If a script attempts to use such an access point
nothing will happen. If the physical attributes are read off
the instrument at such a point, they will all return values of zero.

Note also that in the case of the string only the $x$ coordinate needs
to be specified.

\section{Instrument Visualisations}
\label{instrument_visualisations}
\hierindex{instrument!visualisation facility}
\tao\ is capable of producing visualisations of the instruments constructed
when a script is invoked. We have already seen one example of this is figure
\ref{fig:circle_example}, but in that example the instrument only showed a 
single circular sheet of material. Figure \ref{fig:compound_example} shows a
\Term{compound instrument} consisting of five strings attached to a
rectangular sheet. Access points are marked by small red points on the image
and in this case they show that the left hand sides of each string are
coupled to points on the rectangular sheet.

\begin{figure}[htb]
  \begin{Label}{fig:compound_example}
    \begin{center}
    \Image{compound_example}{height=8cm}{gif}
    \end{center}
    \caption{Screenshot of a compound instrument}
  \end{Label}
\end{figure}

It is possible to translate, zoom, and rotate the image by holding down the
left, middle and right mouse buttons respectively and moving the mouse.
\tao\ automatically labels instruments and devices with their names
(the names they are given when they are created in a script) and it is
possible to toggle both types of labels on and off in cases where the
graphics window becomes too cluttered with information. This is achieved
by pressing the \textbf{I} key to toggle instrument labels on and off,
and the \textbf{D} key to toggle device labels.

\section{Parameters}
\label{section:parameters}
The term \Term{parameter}\/ is used as a generic term to refer to all
numerical variables in \tao. There are floating point and integer parameters
and the latter are declared as either being of type \verb|Integer|,
\verb|Counter| or \verb|Flag|. These types are only used to make the
intended function of a particular integer variable clear to the human reader.
As far as the system is concerned they are functionally identical. 

\section{Pitches and Frequencies}
\label{section:pitches_and_frequencies}
The Pitch\index{Pitch} object provides a generalised mechanism for expressing and
converting between various pitch and frequency formats. The formats
supported are as follows:

\begin{itemize}
\item \texttt{\emph{value} Hz} (cycles per second, analogous to Csound's cps notation);
\item \texttt{\emph{octave}.\emph{semitone}} (analogous to Csound's pch notation);
\item \texttt{\emph{octave}.\emph{decimal}} (analogous to Csound's oct notation);
\item note name notation (the pitch is represented as a character string).
\end{itemize}

In the last case the pitch names \verb|C-G| can be used followed by an
optional \verb|b| for flat or \verb|#| for sharp. The basic pitch
name of the note is then followed by an octave number (whose value
has the same meaning as the integer parts of the \verb|pch| and
\verb|oct| notations).
Finally an optional microtonal adjustment may be added in the form of a
fraction \verb|+<x>/<y>| or \verb|-<x>/<y>|
which adds/subtracts a fraction of a semitone to/from the pitch given.
Some practical examples are given below:

\begin{verbatim}
    Pitch p1 = 110.5 Hz;
    Pitch p2 = C#5+1/2;
    Pitch p3 = 8.05 pch;
    Pitch p4 = 6.1764 oct;
\end{verbatim}

Pitch methods include:

\begin{description}
\item[asPitch()]
returns a number representing the pitch converted to pch notation;
\item[asOctave()]
returns a number representing the pitch converted to oct notation;
\item[asFrequency()]
returns a number representing the pitch converted to a frequency;
\item[asName()]
returns a character string representing the name of the pitch.
\end{description}
